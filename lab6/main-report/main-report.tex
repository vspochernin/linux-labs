\documentclass[a4paper]{article}
\usepackage[margin=3cm]{geometry}
\usepackage[utf8]{inputenc}
\usepackage[russian]{babel}
\usepackage{pgfplots}
\usepackage{minted}
\title{Отчёт по лабораторной работе №6\\<<Конфигурация и установка ядра Linux>>}
\author{
    Почернин Владислав 3530904/00004
    \and
    Разукрантов Владислав 3530904/00004
    \and
    Луцай Павел 3530904/00004
}
\begin{document}
    \maketitle


    \section{Цель}
    \begin{itemize}
        \item Получить собранное и запущенное на системе ядро-Linux, при это
        подсчитать и сравнить время сборки ядра при различном числе потоков сборки
    \end{itemize}


    \section{Задачи}
    \begin{enumerate}
        \item Загрузка исходного кода ядра
        \item Сборка ядра
        \item Установка ядра
        \item Проверка правильности установки
        \item Подсчёт времени сборки при различном числе потоков сборки
    \end{enumerate}


    \section{Аппаратная платформа}
    \begin{itemize}
        \item Intel(R) Core (TM) i3-6100U CPU @ 2.30GHz
    \end{itemize}


    \section{Программная платформа}
    \begin{itemize}
        \item ОС Manjaro Linux, ядро:
        \begin{minted}[frame=single]{text}
Linux manjaro5.13.19-2-MANJARO #1 SMP PREEMPT Sun Sep 19 21:31:53 UTC 2021
x86_64 GNU/Linux
        \end{minted}
        \item Программы GNU
    \end{itemize}


    \section{Основная часть}

    \subsection{Загрузка исходного кода ядра}
    \begin{itemize}
        \item Были установлены исходные коды ядра
        \begin{minted}[frame=single]{text}
$ $wget https://cdn.kernel.org/pub/linux/kernel/v5.x/linux-5.14.14.tar.xz
        \end{minted}
        \item А далее распакованы
        \begin{minted}[frame=single]{text}
$ $tar -xvf linux-5.14.14.tar
        \end{minted}
    \end{itemize}

    \subsection{Сборка ядра}
    \begin{itemize}
        \item Был использован nconfig для конфигурации на основе текущей конфигурации
        \begin{minted}[frame=single]{text}
$ zcat /proc/config.gz > .config
$ make nconfig
        \end{minted}
        \item Сборка ядра
        \begin{minted}[frame=single]{text}
$ make -j4
        \end{minted}
        \item Сборка модулей
        \begin{minted}[frame=single]{text}
$ make modules -j4
        \end{minted}
    \end{itemize}

    \subsection{Установка ядра}
    \begin{itemize}
        \item Готовые файлы ядра были скопированы в /boot с помощью
        \begin{minted}[frame=single]{text}
$ cp -v arch/x86_64/boot/bzImage /boot/vmlinuz-vlad-x86_648
        \end{minted}
        \item Были установлены модули
        \begin{minted}[frame=single]{text}
$ make modules_install
        \end{minted}
        \item Далее был скопирован существующий и изменен на новый mkinitcpio preset
        \begin{minted}[frame=single]{text}
$ cp /etc/mkinitcpio.d/linux.preset /etc/mkinitcpio.d/linuxvlad.preset
$ vim /etc/mkinitcpio.d/linuxvlad.preset
        \end{minted}
        Новый пресет:
        \begin{minted}[frame=single]{text}
# mkinitcpio preset file for the 'linux513' package

ALL_config="/etc/mkinitcpio.conf"
ALL_kver="/boot/vmlinuz-vlad-x86_64"

PRESETS=('default' 'fallback')

#default_config="/etc/mkinitcpio.conf"
default_image="/boot/initramfs-vlad-x86_64.img"
#default_options=""

#fallback_config="/etc/mkinitcpio.conf"
fallback_image="/boot/initramfs-vlad-x86_64-fallback.img"
fallback_options="-S autodetect"
        \end{minted}
        \item Был сгенерирован initframs-образ для новго ядра
        \begin{minted}[frame=single]{text}
$ mkinitcpio -p linuxvlad
        \end{minted}
        \item Был обновлен grub
        \begin{minted}[frame=single]{text}
$ sudo update-grub
        \end{minted}
    \end{itemize}

    \subsection{Проверка правильности установки}
    \begin{itemize}
        \item После перезагрузки системы была выбрана нужная запись (ядро) при загрузке и было проверено,
        что нужное ядро загрузилось
        \begin{minted}[frame=single]{text}
$ uname -a
Linux manjaro 5.14.14-MANJAROVLAD #4 SMP PREEMPT Mon Nov 8 15:27:27
MSK 2021 x86_64 GNU/Linux
        \end{minted}
    \end{itemize}

    \subsection{Подсчет времени сборки при различном числе потоков сборки}
    \begin{itemize}
        \item С помощью сценария была запущена сборка при числе потоков
        1-9 и было засечено время
        \begin{minted}[frame=single]{bash}
#!/bin/bash

for n in {1..17}; do
  make clean > /dev/null
  echo потоков: $n
  time make -j$n > /dev/null
done
        \end{minted}
        \item Были получены следующие результаты работы сценария и построен график
        \begin{minted}[frame=single]{text}
потоков: 1

real 32m8,632s
user 29m14,678s
sys 2m47,105s
потоков: 2

real 16m49,228s
user 30m27,384s
sys 2m54,495s
потоков: 3

real 14m55,926s
user 40m21,668s
sys 3m38,410s
потоков: 4

real 13m44,206s
user 47m41,521s
sys 4m10,400s
потоков: 5

real 13m45,481s
user 47m55,023s
sys 4m2,264s
потоков: 6

real 13m46,710s
user 48m8,534s
sys 3m55,223s
потоков: 7

real 13m50,448s
user 48m15,954s
sys 3m50,460s
потоков: 8

real 13m48,022s
user 48m19,709s
sys 3m49,043s
потоков: 9

real 13m48,751s
user 48m22,022s
sys 3m48,756s
        \end{minted}

        \begin{tikzpicture}
            \begin{axis}[
                height = 300pt,
                width = 340pt,
                xbar,
                ytick = data,
                nodes near coords,
                y dir = reverse,
                xmin = 0,
                xmax = 1800,
                xlabel = {Время сборки, секунды},
                ylabel = {Число потоков сборки},
            ]
                \addplot coordinates {
                    (1929,1)
                    (1009,2)
                    (896,3)
                    (824,4)
                    (824,4)
                    (825,5)
                    (827,6)
                    (830,7)
                    (828,8)
                    (829,9)
                };
            \end{axis}
        \end{tikzpicture}
        \item Изначально, время сборки уменьшалось при увеличении количества потоков.
        Но при числе потоков > 4 время сборки перестало снижаться. Из этого следует, что оптимальное
        число потоков сборки равно числу потоков используемого процессора
    \end{itemize}


    \section{Выводы}
    \begin{itemize}
        \item Было собрано и запущено ядро, для чего был загружен исходный код ядра,
        произведена его сборка и установка, проверена правильность установки и
        подсчитано время сборки при различном числе потоков сборки
        \item Одной из возникших трудностей было перемещение скомпилированного ядра в boot, мы
        предполагали, что это должна сделать команда make install, но все решилось
        тем, что образ был перемещен вручную
        \item На данной системе с четырехпоточным процессором оптимальное число потоков сборки - 4
    \end{itemize}

\end{document}
